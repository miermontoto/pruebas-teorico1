\chapter{Preguntas del enunciado (ejercicio 5)}
\section{Posibles consecuencias ante un fallo}
El fallo de un software de estas características, ya sea por un error en el cálculo de los impuestos, caída
del servicio o cualquier otro problema que afecte al funcionamiento del sistema, puede tener consecuencias
catastróficas para los ciudadanos, ya que el cálculo de los impuestos es un proceso que afecta directamente
a la economía tanto del individuo como del estado.

En caso de fallo, la ciudadanía podría verse afectada por sanciones económicas por parte de la administración
sin justificación ninguna o incluso por la pérdida de dinero en concepto de impuestos que no se deberían haber
pagado. Por otro lado, el estado podría verse afectado por la pérdida de ingresos en concepto de impuestos que
no se han cobrado, lo que podría afectar a la economía del país.

\section{Noticias relevantes}
El fallo de software relacionado con administraciones públicas no es ficción, y de hecho es
más que frecuente. A continuación se citan algunas noticias que dejan en evidencia este hecho:
\begin{itemize}
	\item <<El 61\% de los usuarios han tenido problemas al usar las webs o apps de administraciones públicas>> \cite{newtral}
	\item <<Fallos en los programas y sistemas caídos: programas desactualizados lastran la actividad en Empleo>> \cite{abc}
	\item <<Sigue sin funcionar>> \cite{elpais}
\end{itemize}
