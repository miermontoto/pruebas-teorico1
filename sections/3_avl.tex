\chapter{Análisis de valores límite}
Para el análisis de valores límite, se han seleccionado los valores límite de las clases de equivalencia
de entrada definidas en el capítulo anterior, poniendo especial atención al cumplimiento de las aperturas
de los intervalos.

\section{Base imponible ($BI$)}
Debido a la naturaleza de esta variable, se han seleccionado los valores límite de los intervalos definidos
anteriormente.
\begin{itemize}
	\item $BI = 0$€
	\item $BI = 8999$€
	\item $BI = 9000$€
	\item $BI = 12449$€
	\item $BI = 12450$€
	\item $BI = 20199$€
	\item $BI = 20200$€
	\item $BI = 35199$€
	\item $BI = 35200$€
	\item $BI = 59999$€
	\item $BI = 60000$€
	\item $BI = 299999$€
	\item $BI = 300000$€
\end{itemize}
\newpage{}
\section{Retenciones ($R$)}
Esta variable es especialmente interesante para el AVL, ya que el cálculo de las retenciones depende
directamente de la base imponible, por lo que debería haber una relación directa entre los valores límite
de ambas variables.
\begin{itemize}
	\item $R = 0$€
	\item $R = BI - 1$€
	\item $R = BI$€
	\item $R = BI + 1$€
\end{itemize}

\section{Número de empleadores}
En esta sección no se considera relevante aplicar técnicas AVL, ya que el mero cumplimiento de las
diferentes clases de equivalencia garantiza el correcto funcionamiento del sistema debido a la pertenencia de
los posibles valores al conjunto entero.

\section{Préstamos hipotecarios}
Puesto que al resultado del cálculo de los impuestos se le aplican las deducciones correspondientes
(en este enunciado solo se considera esta), los valores límite de esta variable solo afectan al
hecho de si se aplica o no la deducción en sí, sin tener en cuenta el efecto de la deducción en el
cálculo final. Para una comprobación más exhaustiva, se deberían considerar valores exactos que
modifiquen el resultado final del cálculo de los impuestos teniendo en cuenta dichas deducciones.

Puesto a que, como se indica en el diseño, la deducción no tiene un punto de corte ($9040$€), sino que
se trata de un valor máximo, el valor del préstamo hipotecario no se considera relevante para el
planteamiento de los casos de prueba ya que tampoco se calcula el valor de la deducción para las pruebas.
